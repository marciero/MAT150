This course is an introduction to the mathematical analysis, probabilistic modeling, and computer simulation of random events and processes. 
{\it Probability}, as a field of mathematics, gives us a framework for describing the variation and randomness present in many phenomena. This framework allows us to create models that mimic real-world random data generating proceses, and to analyze, draw inferences, predict outcomes, and perform experiments with real and fake data.  
This course is focuses on three fronts: theoretical analysis, probabilistic modeling, and computer simulation.
We will study the mathematical underpinnings of continuous and discrete random variables and probability distributions. We will develop and analyze probabilistic and statistical models, and create simulations based on these models.
Simulations play an increasingly important role in the study of probability.  Not only can they give valuable insights into our analytical results, but they can also allow us to obtain results for which exact analytical results are intractible or impossible to obtain. They also allow us  to experiment with different  real world, ``what if'' different scenarios, make predictions, and to ``interrogate'' our models; for example to determine whether they can detect effects present in fake data with known prescribed properties. Simulations